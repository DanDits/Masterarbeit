% Chapter 0

\chapter{Einleitung}
Im Zentrum dieser Arbeit steht die stochastische Klein-Gordon-Gleichung (sKGG)
\begin{align*}
\dtt{u}(t,x,y)&=\alpha(y) \Laplace u(t,x,y) - \beta(x,\omega)u(t,x,y), \: t>0, \, x\in \Torus^d\\
u(0,x,y)&=u_0(x,y), \: x\in \Torus^d\\
\dt{u}(0,x,y)&=v_0(x,y), \: x\in \Torus^d.
\end{align*}
Diese ist eine Verallgemeinerung der klassischen Klein-Gordon-Gleichung mit periodischen Randbedingungen. Die Parameter $\alpha$ und $\beta$, die Anfangswerte $u_0$ und $v_0$, und somit auch die Lösung $u$, erhalten eine zusätzliche Abhängigkeit von einem Vektor $y\in\R^N$. Dieser Vektor $y$ stellt eine Realisierung einer $N$-dimensionalen Zufallsvariable dar und wir gewinnen die Möglichkeit, Unsicherheit für die Wahl der Modellparameter direkt in die partielle Differentialgleichung zu integrieren. Somit können als einfaches Beispiel physikalische Messfehler durch eine Normalverteilung modelliert werden.\\[0.2cm]
Da die Lösung der sKGG eine stochastische Größe ist, versucht man mit der \emph{uncertainty quantification} gewisse stochastische Kenngrößen der Lösung zu erhalten. Wir werden uns dabei auf den Erwartungswert und die Varianz beschränken, die Berechnung höherer Momente oder von Verteilungsfunktionen ist jedoch ebenfalls ohne größeren Mehraufwand möglich.\\In dieser Arbeit werden wir anhand von konkreten Implementierungen verschiedene Ansätze und Verfahren miteinander vergleichen. Das Ziel dieser Arbeit ist ein Vergleich der Güte und des Aufwands verschiedener Ansätze zur Berechnung gewisser statistischer Kenngrößer der zufallsabhängigen Lösung. Die vorhandenen Pseudocodes orientieren sich am Syntax von \emph{Python}, so definiert beispielsweise die Einrücktiefe einen zusammenhängenden Codeblock.\\
Eine der Hauptschwierigkeiten beim Lösen der stochastischen KGG ist es für hochdimensionale Zufallsräume mit $N>>1$ den zunehmend bemerkbaren \emph{Fluch der Dimensionalität} in den Griff zu bekommen. Dies betrifft vor allem Kollokationsverfahren, die ein Gitter aus dem stochastischen Raum bilden und für jeden Gitterpunkt die Lösung einer deterministischen KGG benötigen. Da die Berechnung dieser punktweisen Lösungen bereits relativ zeitaufwändig ist, wird der Zeitaufwand für diesen Ansatz sehr schnell sehr groß. Bei der Monte-Carlo-Methode, die lediglich auf stochastischer Konvergenz basiert, tritt dieser Fluch der Dimensionalität nicht auf, jedoch ist die stochastische Konvergenzordnung sehr gering und die Konvergenz entsprechend langsam.\\
Ein weiteres Verfahren basiert auf einem Galerkin-Ansatz, welcher die stochastische KGG in ein höherdimensionales Problem transformiert. Eine große Schwierigkeit liegt hierbei darin, die auftretende Differenzialgleichung zu entkoppeln und mit einem Splitting-Ansatz effizient numerisch zu lösen.\\[0.2cm]
In Kapitel \ref{Chapter1} werden die Grundlagen zur Existenztheorie der deterministischen KGG geklärt, das grundlegende Prinzip des Operatorsplittings erläutert, eine kurze Wiederholung der benötigten polynomialen Approximationstheorie präsentiert und stochastische Grundbegriffe zur Modellierung der stochastischen KGG genannt.\\
In Kapitel \ref{Chapter2} präsentieren wir mit der Monte-Carlo-Methode eine erste einfache Möglichkeit den Erwartungswert und die Varianz der Lösung der sKGG zu berechnen. Wir werden feststellen, dass dies sehr rechenaufwändig ist und auf der Evaluierung von vielen deterministischen Einzelproblemen basiert.\\
Als Vorbereitung für die kommenden Lösungsansätze stellt Kapitel \ref{Chapter3} das General Polynomial-Chaos vor und erklärt, wie man aus diesen Polynomdarstellungen die benötigten stochastischen Kenngrößen gewinnen kann.\\
Ebenfalls auf der Evaluierung von vielen deterministischen Einzelproblemen basieren die Kollokationsverfahren in Kapitel \ref{Chapter4}. Dort werden wir einerseits ein auf Interpolation und andererseits ein auf diskreter Projektion basierendes Verfahren kennen lernen.\\
Zuletzt stellt Kapitel \ref{Chapter5} die stochastische Galerkin-Methode vor, welche den klassischen Galerkin-Ansatz auf unseren stochastischen Fall überträgt.