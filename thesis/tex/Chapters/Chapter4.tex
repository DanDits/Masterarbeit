% Chapter 4

\chapter{Stochastische Collocation}
Wie bereits die Monte-Carlo-Methode basieren auch stochastische Collocationsverfahren auf \emph{sampling}, d.h. der wiederholten Auswertung von vielen Zufallsexperimenten. Bei der Monte-Carlo-Methode werden die Punkte dabei abhängig von der gegebenen Verteilung unabhängig und zufällig gewählt. Die Konvergenz ist in einem stochastischen Sinne durch das Gesetz der großen Zahlen gegeben.\\
Dies ist ein großer Unterschied zu den hier vorgestellten stochastischen Collocationsverfahren. Für diese werden die Collocationspunkte nicht zufällig, sondern so gewählt, dass mithilfe von ihnen eine Polynominterpolation oder Quadratur möglich ist. Konvergenz ist dann abhängig davon, wie gut sich die approximierte stochastische Größe durch Polynome von Zufallsvariablen approximieren lässt.\\
Die beiden Ansätze \emph{Interpolation} und \emph{diskrete Projektion} (auch \emph{pseudospektraler Ansatz} genannt) werden für die stochastische KGG im Folgenden detailliert vorgestellt und miteinander verglichen.
\section*{Collocation für die stochastische KGG}
Die stochastische KGG (\ref{skgg}) für $d$ räumliche Dimensionen, $N$ stochastische Dimensionen und $y\in\R^N$ war definiert durch
\begin{align*}
\dtt{u}(t,x,y)&=\alpha(y) \Laplace_x u(t,x,y) - \beta(x,\omega)u(t,x,y), \: t>0, \, x\in \Torus^d\\
u(0,x,y)&=u_0(x,y), \: x\in \Torus^d\\
\dt{u}(0,x,y)&=v_0(x,y), \: x\in \Torus^d
\end{align*}
Im klassischen Sinne bedeutet Collocation, dass eine Approximation für eine diskrete Knotenmenge $\Theta_S=\lbrace y_i\in\R^N\mid i=1,\dots,S\rbrace$ exakt ist. Auf die stochastische KGG übertragen, muss also für $y=y_i\in\Theta_S$ die Lösung $u(t,x,y_i)$ für $x\in\Torus^d$ und $t>0$ vorliegen und die Approximation $w$ erfüllt $w(t,x,y_i)=u(t,x,y_i)$ für $i=1,\dots,S$. Da wir die exakte Lösung $u(t,x,y_i)$ nicht kennen, sondern für einen Zeitpunkt $T>0$ und eine Knotenmenge $\lbrace x_j\in\Torus^d\mid j=1,\dots,H\rbrace$ nur mithilfe des Strang-Splittings approximieren können, gilt also für festes $x_j$ die Collocationsbedingung lediglich approximativ:
\[w(T,x_j,y_i)=u(T,x_j,y_i)\approx \tilde{u}(T,x_j,y_i),\quad i=1,\dots,S\]
Wir wählen daher $\tau$ so klein, dass dieser Fehler im Vergleich zum eigentlichen Collocationsfehler vernachlässigbar gering ist. Zur Vereinfachung der Notation nehmen wir im Folgenden an, die Lösung $u(T,x_j,y_i)$ liege exakt vor.

\section{Interpolation}
Sei $Y$ ein $N$-dimensionaler Zufallsvektor mit stochastisch unabhängigen Komponenten und $\Theta_S\subset I_Y$ eine zu den Verteilungen von $Y_i$ passende $S$ elementige Knotenmenge. Seien weiter $\lbrace \Phi_m \mid |m|\le P \rbrace$ die zu den Verteilungen passenden gPC Basisfunktionen und im Folgenden mit $m=0,\dots M=\binom{N+P}{P}-1$ durchnummeriert. Eine Diskussion zur Nummerierung findet man im vorherigen Kapitel.\\
Dann wählen wir als Interpolationspolynom die Funktion
\begin{equation}
\label{eqn:coll_interpol_basic}
w_M^{(j)}(Y)=\sum_{m=0}^M\hat{w}_m^{(j)}\Phi_m(Y)\in\Poly_P^N(Y),\quad j=1,\dots,H
\end{equation}
und fordern als Interpolationsbedingung
\[w_M^{(j)}(y_i)=u(T,x_j,y_i),\quad i=1,\dots,S, j=1,\dots,H\]
Die Notation macht klar, dass wir zu einem gegebenen Zeitpunkt $T>0$ und jedem festen Knotenpunkt $x_j$ die stochastische Funktion $u(T,x_j,Y)$ durch ein Polynom in $Y$ approximieren wollen.\\
Betrachten wir die Gleichung (\ref{eqn:coll_interpol_basic}) in den Knotenpunkten $y_i$ für jedes $i=1,\dots,S$, so können wir das entstehende System kompakt schreiben als
\begin{equation}
\label{eqn:interpol_compact}
A\hat{w}=u
\end{equation}
mit 
\begin{align*}
A&=(\Phi_m(y_i))_{i=1,\dots,S;m=0,\dots,M}\in\R^{S\times M+1}\\
\hat{w}&=(\hat{w}_m^{(j)})^T_{m=0,\dots,M;j=1,\dots,H}\in\R^{M+1\times H}\\
u&=(u(T,x_j,y_i)^T_{i=1,\dots,S;j=1,\dots,H}\in\R^{S\times H}\\
\end{align*}
Für jeden Knotenpunkt $x_j$ ist also ein lineares Gleichungssystem der Größe $S\times M+1$ zu lösen. Die obige Schreibweise erlaubt es in vielen Programmiersprachen die Systeme gleichzeitig zu lösen und bietet somit einen starken Laufzeitvorteil gegenüber dem Lösen von $H$ separaten Gleichungen. Um sicherzustellen, dass das System nicht unterbestimmt ist, muss gelten $S\ge M+1=\binom{N+P}{P}$. Ist $S\neq M+1$, so verwenden wir die Pseudoinverse von $A$. Später werden wir an verschiedenen Beispielen Konsequenzen dieser Bedingung genauer betrachten.
\begin{mathbem}
In Gleichung (\ref{eqn:gpc_approx_exp}) wurde gezeigt, dass sich der Erwartungswert an der Stelle $x_j$ approximieren lässt durch
\[\E[u(t,x_j,Y)]\approx \hat{w}_0^{(j)}\]
Da die Gleichungen gekoppelt sind, müssen aber trotzdem auch alle anderen $\hat{w}_m$ berechnet werden.
\end{mathbem}
\subsection{Wahl der Interpolationspunkte}
Wir wollen direkt am Anfang darauf hinweisen, dass im mehrdimensionalen für $N>1$ die Wahl der Knotenpunkte für die Interpolation ein nicht vollständig verstandenes Problem darstellt. Das Konzept der Lagrange-Interpolation, die im eindimensionalen für jede beliebige Knotenmenge möglich ist, lässt sich wie in \autocite{San07} gezeigt konzeptuell auf den mehrdimensionalen Fall erweitern. Allerdings benötigt sie die entsprechende Bedingung $\det(A)\neq 0$, welche besagt, dass die Interpolation durch die gegebenen Knotenpunkte eindeutig ist. Die Beziehung dieser Bedingung zur Wahl der Knotenpunkte stellt ein "'komplexes Problem"' dar.
\subsubsection*{Grundlagen der eindimensionalen Interpolation}
Der Satz von Cauchy ist der erste Orientierungspunkt für die Abhängigkeit der Interpolationsgüte zur Wahl der Knotenpunkte für den eindimensionalen Fall und einem kompakten Intervall.
\begin{maththeorem}[Cauchy]
Sei $f\in C^{S}[-1,1]$ eine $S$ mal stetig differenzierbare Funktion. Dann ist für jede Knotenmenge $\lbrace y_1,\dots,y_{S}\rbrace$ und $y\in[-1,1]$ der Interpolationsfehler gegeben durch
\[f(y)-\mathcal{Q}_{S}f=\frac{f^{(S)}(\xi)}{S!}\prod_{i=1}^{S}(y-y_i),\quad \xi\in [-1,1]\]
\end{maththeorem}

Da die Funktion gegeben ist, hat man keinen Einfluss auf die Ableitung $f^{(S)}$. Wegen
\[\norm{f(y)-\mathcal{Q}_{S}f}\le \frac{\norm{f^{(S)}}_\infty}{S!}\underbrace{\norm{\prod_{i=1}^{S}(y-y_i)}}_{=\norm{w(y)}}\]
ist es somit das Ziel, den Term $\norm{w(y)}$ abhängig von der Norm zu minimieren.
\begin{mathbem}[\chebyspace Interpolation]
Die Nullstellen $y_i=\cos\left(\frac{2i-1}{2S+2}\right)$, für $i=1,\dots,S+1$, der \chebyspace Polynome $T_{S+1}(y)=\cos((S+1)\arccos(y))$ minimieren den Ausdruck $\norm{w(y)}_\infty$ auf $[-1,1]$ und bieten auf $[-1,1]$ optimale Interpolationsgüte. Es gilt 
\[\norm{f(y)-\mathcal{Q}_{S+1}f}_\infty\le \frac{\norm{f^{(S+1)}}_\infty}{2^S(S+1)!}\]
\end{mathbem}
Die \chebyspace Polynome bilden eine Orthogonalbasis bezüglich $\langle \cdot,\cdot\rangle_{L_w^2[-1,1]}$ mit $w(y)=(1-y^2)^{-\onehalf}$. Eine Idee ist es nun, auch Nullstellen anderer orthogonalen Polynombasen als Knotenpunkte zu verwenden um ähnliche Approximationsergebnisse auf nicht-kompakten Intervallen zu erhalten.
\begin{maththeorem}
\label{th:interpol_and_proj}
Sei $\lbrace \Phi_m\in\Poly_m\mid m=0,\dots,M\rbrace\subset L_w^2(I)$ eine Orthonormalbasis von Polynomen und $(y_i,\alpha_i)_{i=0,\dots,M}$ eine Quadraturformel mit Exaktheitsgrad $2M$ bezüglich $w$ und $y_i\neq y_j,i\neq j$, d.h. \[\int_I p(y)w(y)dy=\sum_{i=0}^M\alpha_ip(y_i),\quad p\in\Poly_{2M}\]
Sei weiter das Interpolationspolynom  $\mathcal{Q}_Mf$ einer Funktion $f\colon I\to\R$ bezüglich den Interpolationspunkten $\lbrace y_i\mid i=0,\dots,M\rbrace$ beschrieben durch
\[\mathcal{Q}_Mf(y)=\sum_{m=0}^M\hat{w}_m\Phi_m(y)\]
Die Koeffizienten $\hat{w}_m$ werden in Analogie zu (\ref{eqn:interpol_compact}) berechnet durch das Lösen des linearen Gleichungssystems
\[A\hat{w}=\hat{f}\]
Dann ist \[\hat{w}_i=\sum_{j=0}^M\Phi_i(y_j)\alpha_jf(y_j)\approx \int_I \Phi_i(y)f(y)w(y)dy,\quad i=0,\dots,M\]
\end{maththeorem}
\begin{proof}
Es ist $A_{ij}=\Phi_j(y_i)$. Da wir uns im eindimensionalen Fall befinden ($N$=1) ist die Vandermonde-artige Matrix $A$ invertierbar, da die $\Phi_i$ eine Basis bilden und die $y_i$ paarweise verschieden sind. Es gilt für $\widetilde{w}_i\coloneqq \sum_{j=0}^M\Phi_i(y_j)\alpha_jf(y_j)$ und $\widetilde{w}\coloneqq (\widetilde{w}_i)^T$
\begin{align*}
\left(A\widetilde{w}\right)_i&=\sum_{j=0}^M\Phi_j(y_i)\widetilde{w}_j\\
&=\sum_{j=0}^M\Phi_j(y_i)\sum_{k=0}^M\Phi_j(y_k)\alpha_kf(y_k),\quad \text{Interpol.:} f(y_k)=\sum_{\ell=0}^M\hat{w}_\ell\Phi_\ell(y_k)\\
&=\sum_{\ell=0}^M\hat{w}_\ell \sum_{j=0}^M\Phi_j(y_i)\sum_{k=0}^M\Phi_j(y_k)\alpha_k\Phi_\ell(y_k)\\
&\stackrel{\text{deg}(\Phi_j\Phi_\ell)\le 2M}{=}\sum_{\ell=0}^M\hat{w}_\ell\sum_{j=0}^M\Phi_j(y_i)\underbrace{\int_I\Phi_j(y)\Phi_\ell(y)w(y)dy}_{=\delta_{j\ell}}\\
&=\sum_{\ell=0}^M\hat{w}_\ell\Phi_\ell(y_i)=f(y_i)=\hat{f}_i
\end{align*}
Also gilt $A\widetilde{w}=\hat{f}=A\hat{w}$ und da $A$ regulär ist somit $\widetilde{w}=\hat{w}$.
\end{proof}
\begin{mathbem}
Wir wollen nun dem technischen Satz \ref{th:interpol_and_proj} Leben einhauchen und seine Relevanz verdeutlichen.
\begin{itemize}
\item
Die vorausgesetzte Existenz einer Quadraturformel aus $M+1$ verschiedenen Quadraturpunkten und Exaktheitsgrad $2M$ ist durch die später vorgestellte Gauss-Quadratur erfüllt. Die Quadraturpunkte sind dabei die Nullstellen des $M+1$-ten Basispolynoms und daher paarweise verschieden. Die Quadraturgewichte lassen sich über die erste Komponente dessen Einheitsvektors bestimmen, der beim Golub-Welsch-Algorithmus auch die entsprechende Nullstelle des Polynoms berechnet.
\item
Wir sehen, dass sich im eindimensionalen bei bekannter Quadraturformel die Interpolationskoeffizienten $\hat{w}_m$ auch einzeln ohne Lösen eines linearen Gleichungssystems berechnen lassen. Diese Darstellung führt auf den später vorgestellten zweiten Ansatz zur stochastischen Collocation.
\item
Die Darstellung $\hat{w}_i=\sum_{j=0}^M\Phi_i(y_j)\alpha_jf(y_j)$ kann als Approximation an die Koeffizienten $\int_I \Phi_i(y)f(y)w(y)dy$ der Bestapproximation $P_Mf$ in $L_w^2(I)$ aufgefasst werden. Mithilfe der Dreiecksungleichung lässt sich dann der Fehler der Interpolation beschreiben durch
\begin{align*}
&\norm{\mathcal{Q}_Mf-f}_{L_w^2}\le \underbrace{\norm{P_Mf-f}_{L_w^2}}_{\text{Fehler der Bestapproximation}}+\underbrace{\norm{\mathcal{Q}_Mf-P_Mf}_{L_w^2}}_{\text{Aliasing Fehler}}\\
&=\norm{\mathcal{Q}_Mf-P_Mf}_{L_w^2}+\norm{\sum_{m=0}^M\left(\sum_{j=0}^M\Phi_m(y_j)\alpha_jf(y_j)\right)\Phi_m-\sum_{m=0}^M\left(\int_I\Phi_m(y)f(y)w(y)dy\right)\Phi_m}_{L_w^2}\\
&\stackrel{\text{Parseval}}{=}\norm{\mathcal{Q}_Mf-P_Mf}_{L_w^2}+\sqrt{\sum_{m=0}^M\left(\sum_{j=0}^M\Phi_m(y_j)\alpha_jf(y_j)-\int_I\Phi_m(y)f(y)w(y)dy\right)^2}
\end{align*}
Schlussendlich lässt sich also der Fehler der Interpolation durch den Fehler der Bestapproximation in $L_w^2$ und einen Quadraturfehler bezüglich der Gewichtsfunktion $w$ beschreiben. Für den Fehler der Bestapproximation erwarten wir nach Xiu spektrale Konvergenz. Finden wir eine zur Gewichtsfunktion $w$ passende Quadraturformel, so hat lediglich die Glattheit von $f$ Einfluss auf den Quadraturfehler, da $\Phi_m$ als Polynom bei hinreichend hoher Ordnung exakt integriert wird.
\end{itemize}
\end{mathbem}

\section{Diskrete Projektion}
\subsection{Gauss-Quadratur im eindimensionalen}

\subsection{Mehrdimensionale Quadratur}
\subsubsection{Tensorkonstruktion}
\subsubsection{Smoylaks dünne Gitter}

\todo[inline]{Smolyak Konstruktion in Kapitel drei erklären? (für Punkte Gitter wie bei Interpol benötigt aber auch für allgemeine Operatoren?)}
\todo[inline]{1d interpolation verstanden (gauß interpol, andere auch möglich und gut); mehrdim aber problematisch: wie punkte wählen (tensor, andere ansätze wie smolyak sparse grid)} 

\todo[inline]{Interpolation;Quadratur (Xiu Kap7) mit Gauß, Glenshaw-Curtis; smolyak sparse grids (fully nested failed, weakly nested and non nested; discrete projection;Äquivalenz von Interpol + diskr proj (zumindest mal in 1d, mehr bei entspr nodes/weights?)}
