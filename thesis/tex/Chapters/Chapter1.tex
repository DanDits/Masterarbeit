% Chapter 1

\chapter{Einführung}
In diesem Kapitel werden wir die zu untersuchende Klein-Gordon-Gleichung (KGG) vorstellen und uns mit den zugehörigen häufig verwendeten Parametern vertraut machen. Anschließend gewinnen wir einen numerischen Lösungsansatz aus der Idee des Operatorsplittings. Um dann später die Zufallsabhängigkeit einführen zu können, benötigen wir die Theorie des General Polynomial Chaos. Die in dieser Arbeit untersuchte \emph{uncertainty} in Gestalt von Zufallsvariablen wird allerdings erst in den folgenden Kapiteln in die KGG einfließen.
\label{Chapter1}

\section{Die Klein-Gordon-Gleichung}
Die für diese Arbeit grundlegende Gleichung ist die lineare Klein-Gordon-Gleichung in der Form
\begin{align}
\label{kgg}
\dtt{u}(t,x)&=\alpha \Laplace_x u(t,x) - \beta(x)u(t,x), \: t>0, \, x\in \Torus^d\\
u(0,x)&=u_0(x), \: x\in \Torus^d\\
\dt{u}(0,x)&=v_0(x), \: x\in \Torus^d
\end{align}
Im Folgenden werden wir auf den, hier zur Betonung verwendeten, Index $x$ des Laplace Operators $\Laplace_x=\Laplace$ verzichten.\\
Bevor wir Abhängigkeiten von einer Zufallsvariablen hinzufügen, ist es hilfreich, die Gleichung für deterministische Parameter besser zu verstehen. Diese Mühe ist nicht vergebens, da einige numerische Verfahren zur Bestimmung der Lösung der zufallsabhängigen Gleichung stark auf einem robusten und schnellen Löser der deterministischen Gleichung basieren. Dazu seien als Stichworte Monte-Carlo-Verfahren und Collocations-Verfahren genannt, die wir im späteren Verlauf genauer betrachten werden.
\subsection{(Physikalische) Definitionen}
Die KGG ist eine relativistische Feldgleichung, welche die Kinematik von spin-freien Teilchen, wie dem Pion, beschreibt. Auch das 2012 entdeckte Higgs-Boson ist ein spin-freies Teilchen, es ist jedoch noch unklar, ob es wirklich das von dem Standardmodell der Teilchenphysik vorgesagte ist und sich vom Standardmodell beschreiben lässt \autocite{cern2016}.\\
Dabei ist
\begin{itemize}
\item $\Torus=\R/(2\pi\Z)$ der skalierte eindimensionale Torus. Wir werden ab sofort die vereinfachte Darstellung $\Torus^d=(-\pi,\pi)^d$ mit periodischen Randbedingungen verwenden. Diese Skalierung des Torus ermöglicht das direkte Verwenden der schnellen Fouriertransformation ohne weitere Skalierung, ist jedoch keine Beschränkung der Allgemeinheit. 
\item die periodische Randbedingung eine vereinfachte Beschreibung des Verhaltens der Lösung ohne Einflüsse von Rändern. Solche Einflüsse, wie sie beispielsweise bei Dirichlet- oder Neumann-Randwerten auftreten, werden vernachlässigt und die Lösung als auf einem großen Träger lebend betrachtet.
\item $\absolute{u(t,x)}^2$ physikalisch als Ladungsdichte des Teilchens zum Zeitpunkt $t$ und Ort $x$ interpretierbar \autocite{kleingordon2016}. 
\item $\alpha>0$ das Quadrat der Wellengeschwindigkeit.
\item $\beta(x)>0, \forall x\in \Torus^d,$ in der physikalischen Interpretation das Quadrat aus einer Kombination von Wellengeschwindigkeit, Masse und planckschem Wirkungsquantum. Die Abhängigkeit des Potentialteils der KGG vom Ort $x$ ist dabei eine Verallgemeinerung der klassischen Gleichung.
\end{itemize}
Wir betrachten im Gegensatz zur physikalischen Darstellung nur reellwertige Funktionen $u$, $u_0$ und $v_0$ und fordern (implizit), dass die Anfangswerte $u_0$ und $v_0$ periodisch in $(-\pi,\pi)$ sind. \todo{(Reelle) Existenztheorie?}

\subsection{Exakte Lösungen}
Für spezielle Konfigurationen der Parameter und Anfangswerte können wir exakte Lösungen angeben. Diese sind hilfreich, um die Korrektheit von Implementierungen schnell und zuverlässig testen zu können. Außerdem zeigen sich so schnell die Grenzen und eventuelle Schwächen der Verfahren für gut gestellte Probleme auf.\\[1cm]
Mithilfe des Separationsansatzes $u(t,x)=g(x)f(t)$ ergibt sich aus (\ref{kgg}) 
\begin{equation*}
(\alpha\Laplace g(x)-\beta(x)g(x))f(t) = g(x)f''(t)
\end{equation*}
womit sich aus Lösungen von
\begin{align*}
\alpha\Laplace g(x)-\beta(x)g(x)&=\lambda g(x), \: \lambda\in\R\\
\mu f(t)&=f''(t), \: \mu\in \R
\end{align*}
Lösungen der KGG ergeben. Für $\beta(x) \equiv \beta>\absolute{\lambda}$ ergibt sich das Eigenwertproblem $\Laplace g(x)=\frac{\beta + \lambda}{\alpha}g(x)$ mit periodischen Randbedingungen und eine lineare gewöhnliche Differentialgleichung.\\Die Anfangswerte $u_0$ und $v_0$ der KGG werden dann entsprechend passend gewählt.\\[1cm]
Beispiele für exakte Lösungen ($d=1$) sind
\begin{align*}
u(t,x)&=\sin(\lambda x)(c_1 \sin(\mu t) + c_2 \cos(\mu t)), \quad \mu^2=\beta+\alpha \lambda^2\\ 
u(t,x)&=(c_1 \sin(\lambda x) + c_2 \cos(\lambda x))\sin(\mu t), \quad \mu^2=\beta+\alpha \lambda^2\\
u(t,x)&=\exp(-\cos(x))\sin(\mu t), \quad \beta(x)=\mu^2+\cos(x)+\sin^2(x), \, \mu \in \R
\end{align*}
wobei $\lambda\in \pi\Z$ um die Periodizität zu gewährleisten und $c_1, c_2 \in \R$ frei wählbare Skalierungsfaktoren.\\
Diese lassen sich auch auf höhere Dimensionen erweitern, beispielsweise
\begin{eqnarray}
u(t,x)=\exp(-\cos(x_1+\dots+x_d))\sin(\mu t)\\
\beta(x)=\mu^2+d(\cos(x_1+\dots+x_d)+\sin^2(x_1+\dots+x_d)), \, \mu\in\R \nonumber
\end{eqnarray}

% Visualize exact solutions by giving two example plots
\begin{figure}[!htb]
\minipage{0.5\textwidth}
  \includegraphics[width=\linewidth]{Figures/kgg_exact_solution_example1d.png}
  \caption{Exakte Lösung für $d=1$}
\endminipage
\minipage{0.5\textwidth}
  \includegraphics[width=\linewidth]{Figures/kgg_exact_solution_example2d.png}
  \caption{Exakte Lösung für $d=2$}
\endminipage
\captionsetup{labelformat=empty}
\caption{Lösung zum Zeitpunkt $t=0.5$ für die KGG (\ref{kgg}) mit Parametern $\alpha=1$, $
\beta(x)=4+d(\cos(\sum_{j=1}^dx_j)+\sin^2(\sum_{j=1}^dx_j))$.}
\end{figure}

Weitere --allerdings nicht-periodische-- Lösungen finden sich in \autocite{andreipolyanin2004}.

\section{Operatorsplitting}
Um einen Ansatz für numerische Approximationen an die Lösung der KGG zu erhalten bietet sich ein klassischer Operatorsplitting-Ansatz an. Dabei wiederholen wir an dieser Stelle zuerst kurz die relevante Approximationstheorie und diskutieren mögliche Varianten.
\todo[inline]{Späteres Kapitel: Unsere Anwendung auf KGG; WaveSolver und LinhypSolver; Stabilität ohne CFL für $w=1$}
\subsection{Generelle Idee}
Angenommen wir stehen vor dem Problem die Differentialgleichung 
\begin{equation}
\label{gencompdgl}
\dt{u}(t)=(L+R)u(t), \, t>0, \quad u(0)=u_0
\end{equation}
zu lösen. Dabei ist im einfachsten Fall $u$ eine vektorwertige Funktion und $L$ bzw. $R$ zwei Matrizen.\\
Sind aufgrund der Struktur der Matrizen --beispielsweise könnte $L$ eine untere und $R$ eine obere Dreiecksmatrix sein-- die Gleichungen 
\begin{align*}
\dt{v}(t)&=Lv(t), \quad v(0)=v_0\\
\dt{w}(t)&=Rw(t), \quad w(0)=w_0
\end{align*}
deutlich einfacher zu lösen als die Gleichung (\ref{gencompdgl}), so lässt sich aus den Lösungen $v$ und $w$ dennoch eine Approximation an $u$ gewinnen.\\
Wir \emph{teilen} (engl. split) die Gleichung also in zwei neue Gleichungen auf. Häufig lässt sich dann eine der gewonnen Gleichungen sogar exakt lösen. Dieser Ansatz funktioniert auch für (nicht-lineare) Operatoren anstelle von Matrizen.\\
Diese zentrale Idee werden wir verwenden, um die KGG 
\begin{equation*}
\dtt{u}(t,x)=\alpha\Laplace u(t,x)-\beta(x)u(t,x)
\end{equation*}
zu splitten. Dabei wird der Operator $\alpha\Laplace$ in $L$ einfließen und $-\beta(x)$ in $R$. Man beachte jedoch, dass die zeitliche Ableitung zweiter Ordnung von $u$ eine Umschreibung in ein System erster Ordnung erfordert, bevor ein Splittingansatz sinnvoll ist.

\subsection{Lie-Trotter-Splitting}
\label{seclietrotter}
Wir beginnen mit einer rigerosen Einführung in Splittingverfahren. Als zentrales Ergebnis werden wir dabei das Strang-Splitting erhalten, welches einen Fehler $\mathcal{O}(\deltat^2)$ in Abhängigkeit von der Zeitschrittweite $\deltat$ besitzt.\\
Die folgende Einführung ist angelehnt an \autocite{patrickdiplom}, welche wiederum aus \autocite[Kapitel II.3 bis II.5]{HairerLubichWanner} übernommen wurde.
\begin{mathdef}[Fluss einer Differentialgleichung]
Der Fluss $\varphi_t$ einer reversiblen Differentialgleichung
\begin{equation}
\label{orddgl}
\dt{y}=f(y),\quad y(0)\:\text{ gegeben}
\end{equation}
mit $f\colon \R^d\to\R^d$ ist die injektive Abbildung, die einem gegebenen $y_0$ die exakte Lösung $y(t)$ zum Zeitpunkt $t$ zuordnet, wobei $y(0)=y_0$. Also gilt
\begin{equation*}
\varphi_t(y_0)=y(t)\enspace und\enspace y(0)=y_0
\end{equation*}
\end{mathdef}

Um Splittingverfahren von möglichst höher Ordnung zu bekommen ist es wichtig, die einzelnen Teilverfahren bestmöglich zu kombinieren. Hierbei spielt das adjungierte Verfahren eine wichtige Rolle, da es dabei hilft, symmetrisierte Verfahren mit gerader Ordnung zu gewinnen.
\begin{mathdef}[Adjungiertes Verfahren]
Ist $\Phi_{\deltat}\colon\R^d\to\R^d$ ein Einschrittverfahren so definieren wir das adjungierte Verfahren $\Phi_{\deltat}^*$ als das Inverse des Einschrittverfahrens mit negativer Schrittweite $\deltat$, also \[\Phi_{\deltat}^*\coloneqq\Phi_{-\deltat}^{-1}\]
Das bedeutet, dass $y_1=\Phi_{\deltat}^*(y_0)$ implizit durch $\Phi_{-\deltat}(y_1)=y_0$ definiert wird.\\
Ist $\Phi_{\deltat}^*=\Phi_{\deltat}$ so nennen wir das Verfahren symmetrisch.
\end{mathdef}
Beispielsweise sind das explizite Euler Verfahren $\Phi_{\deltat}^E$ und das implizite Euler Verfahren $\Phi_{\deltat}^{IE}$ zueinander adjungiert: 
\[\Phi_{-\deltat}^E(y_1)=y_0\implies y_1+(-\deltat)f(y_1)=y_0\implies y_1=y_0+\deltat f(y_1)=\Phi_{\deltat}^{IE}(y_0)\] \\
Ohne Beweis sei bemerkt, dass $(\Phi_{\deltat}^*)^*=\Phi_{\deltat}$ und 
\begin{equation}
\label{adjlemma}
(\Phi_{\deltat}\circ \Psi_{\deltat})^*=\Psi_{\deltat}^* \circ \Phi_{\deltat}^*
\end{equation}

\begin{mathdef}[Konsistenzordnung]
Wir sagen, dass eine numerische Methode zum Lösen von (\ref{orddgl}) (konsistent) von Ordnung $p$ ist, wenn der lokale Fehler für hinreichend glattes $y$
\[\Phi_{\deltat}(y(t))=\varphi_{\deltat}(y(t))+\mathcal{O}(\deltat^{p+1})\] erfüllt.
\end{mathdef}
\begin{maththeorem}[Ordnung des adjungierten Verfahrens, vgl. 
{\autocite[Theorem II.3.2]{HairerLubichWanner}}]
\label{adjfloworder}
Sei $\varphi_t$ der exakte Fluss von (\ref{orddgl}) und sei $\Phi_{\deltat}$ ein Einschrittverfahren von Ordnung $p$ welches 
\[\Phi_{\deltat}(y_0)=\varphi_{\deltat}(y_0)+C(y_0)\deltat^{p+1}+\mathcal{O}(\deltat^{p+2})\]
erfüllt. Dann ist das adjungierte Verfahren $\Phi_{\deltat}^*$ ebenfalls von Ordnung $p$ und erfüllt
\[\Phi_{\deltat}^*(y_0)=\varphi_{\deltat}(y_0)+(-1)^{p}C(y_0)\deltat^{p+1}+\mathcal{O}(\deltat^{p+2})\]
Ist insbesondere $\Phi_{\deltat}$ symmetrisch, so ist wegen $C(y_0)=(-1)^pC(y_0)$ die Ordnung $p$ gerade.
\end{maththeorem}
Nun zerlegen wir die ursprüngliche Gleichung (\ref{orddgl}) in zwei Teile
\[\dt{y}=f(y)=A(y)+B(y) \]
Das Aufteilen ist hierbei nicht eindeutig! Eine Möglichkeit im zweidimensionalen wäre zum Beispiel: $A=\text{P}_{(1,0)^T}f,\,B=\text{P}_{(0,1)^T}f$, wobei $\text{P}_v$ die Projektion auf $\text{span}(v)$ darstellt. Häufig --ebenso wie in unserem Fall der KGG-- ist die bereits vorhandene natürliche Aufteilung der Summe die Methode der Wahl.\\
Sind nun $\varphi_{\deltat}^A$ und $\varphi_{\deltat}^B$ die exakten Flüsse der Systeme
\begin{align}
\dt{y}&=A(y)\quad \text{und} \label{splitsimple1}\\ 
\dt{y}&=B(y) \label{splitsimple2}
\end{align}
so gewinnen wir daraus ein Verfahren zur Lösung der ursprünglichen Gleichung.\\
Hierzu starten wir von einem gegebenen Anfangswert $y_0$ und lösen das System (\ref{splitsimple1}) um einen Zwischenwert $y_{\onehalf}$ zu erhalten. Verwenden wir diesen als neuen Startwert für das System (\ref{splitsimple2}), so erhalten wir einen daraus einen Wert $y_1$, welcher eine Approximation an die Lösung $\varphi_{\deltat}(y_0)$ des ursprünglichen Systems ist.\\[1cm]
\noindent\begin{minipage}{0.3\textwidth}
\begin{tikzpicture}
  \matrix (m) [matrix of math nodes,row sep=3em,column sep=4em,minimum width=2em]
  {
     \, & y_1 \\
     y_0 & y_{\onehalf} \\};
  \path[-stealth]
    (m-2-1.east|-m-2-2) edge node [below] {$\varphi_{\deltat}^A$} (m-2-2)
    (m-2-2) edge node [right] {$\varphi_{\deltat}^B$} (m-1-2)
    (m-2-1) edge [dashed] node [above left] {$\Phi_{\deltat}$} (m-1-2);
\end{tikzpicture}
\begin{tikzpicture}
  \matrix (m) [matrix of math nodes,row sep=3em,column sep=4em,minimum width=2em]
  {
     y_{\onehalf} & y_1 \\
     y_0 & \, \\};
  \path[-stealth]
    (m-1-1.east|-m-1-2) edge node [above] {$\varphi_{\deltat}^A$} (m-1-2)
    (m-2-1) edge node [left] {$\varphi_{\deltat}^B$} (m-1-1)
    (m-2-1) edge [dashed] node [below right] {$\Phi_{\deltat}^*$} (m-1-2);
\end{tikzpicture}
\end{minipage}%
\hfill%
\begin{minipage}{0.7\textwidth}
Die Reihenfolge $A\to B$ ist willkürlich und kann auch umgekehrt werden. Tatsächlich sind die resultierenden Verfahren zueinander adjungiert wegen (\ref{adjlemma}) und der Tatsache, dass der exakte Fluss wegen $y_1=\varphi_{-\deltat}^{-1}(y_0)=\varphi_{\deltat}^*(y_0)=\varphi_{\deltat}(y_0)$ als symmetrisches Einschrittverfahren auffassbar ist. Kurz:\\
\begin{align}
\label{liesplittings}
\Phi_{\deltat}=\varphi_{\deltat}^B \circ \varphi_{\deltat}^A\\
\Phi_{\deltat}^*=\varphi_{\deltat}^A \circ \varphi_{\deltat}^B\nonumber
\end{align} 
\end{minipage}

\begin{maththeorem}[Ordnung des Lie-Trotter-Splittings]
\label{lieorder1}
Das Lie-Trotter-Splitting $\Phi_{\deltat}=\varphi_{\deltat}^B \circ \varphi_{\deltat}^A$ und sein adjungiertes $\Phi_{\deltat}^*=\varphi_{\deltat}^A \circ \varphi_{\deltat}^B$ sind von Ordnung $p=1$.
\end{maththeorem}
\begin{proof}
Die Taylorreihe von $\varphi_{\deltat}$ lässt sich wegen 
\[\varphi_{\deltat}(y_0)=y(\deltat)=y(0)+\deltat \underbrace{y^\prime(0)}_{=f(y_0)}
+\frac{\deltat^2}{2}\underbrace{y^{\prime\prime}(0)}_{=f^\prime(y_0)y^\prime(0)=f^\prime(y_0)f(y_0)}+\mathcal{O}(\deltat^3)\] darstellen als 
\[\varphi_{\deltat}(y_0)=y_0+\deltat f(y_0)+\frac{\deltat^2}{2}f^\prime(y_0)f(y_0)+\mathcal{O}(\deltat^3)\]
Diese vergleichen wir nun mit der Taylorreihe des Splittings
\begin{align*}
\left(\varphi_{\deltat}^B \circ \varphi_{\deltat}^A\right)(y_0)&=
\varphi_{\deltat}^B\left(y_0+\deltat f^A(y_0)+\frac{\deltat^2}{2}f^{\prime A}(y_0)f^A(y_0)+\mathcal{O}(\deltat^3)\right)\\
&=\left(y_0+\deltat f^A(y_0)+\frac{\deltat^2}{2}f^{\prime A}(y_0)f^A(y_0)\right)
+\deltat f^B\left(y_0+\deltat f^A(y_0)+\mathcal{O}(\deltat^2)\right)\\
&\quad+\frac{\deltat^2}{2}f^{\prime B}\left(y_0+\mathcal{O}(\deltat)\right)f^B\left(y_0+\mathcal{O}(\deltat)\right)+\mathcal{O}(\deltat^3)\\
&=y_0+\deltat f(y_0)+\frac{\deltat^2}{2}f^\prime(y_0)f(y_0)\\
&\quad+\frac{\deltat^2}{2}\left(f^{\prime B}(y_0)f^A(y_0)-f^{\prime A}(y_0)f^B(y_0)\right)+\mathcal{O}(\deltat^3)
\end{align*}
so bekommen wir für hinreichend glatte Funktionen $f^A$, $f^{\prime A}$, $f^B$ und $f^{\prime B}$ den Fehler \[\varphi_{\deltat}(y_0)-\left(\varphi_{\deltat}^B \circ \varphi_{\deltat}^A\right)(y_0)=\mathcal{O}(\deltat^2)\] und die Ordnung $p=1$. Mit Satz \ref{adjfloworder} ist das adjungierte Verfahren ebenfalls von Ordnung $p=1$.
\end{proof}

\subsection{Strang-Splitting}
\label{secstrang}
Durch geschicktes Kombinieren der Einschrittverfahren $\Phi_{\deltat}$ und $\Phi_{\deltat}^*$ mit verschiedenen Schrittweiten lassen sich Splittingverfahren beliebiger Ordnung konstruieren.
\begin{maththeorem}[vgl. {\autocite[Theorem II.4.1ff]{HairerLubichWanner}}]
Ist $\Phi_{\deltat}$ ein Einschrittverfahren von Ordnung $p$ so ist die Zerlegung
\[\Psi_{\deltat}=\Phi_{a_s\deltat}\circ \Phi_{b_s\deltat}^*\circ\dots\circ\Phi_{b_2\deltat}^*\circ\Phi_{a_1\deltat}\circ\Phi_{b_1\deltat}^*\]
von Ordnung $p+1$, wenn 
\begin{align*}
\sum_{j=1}^sa_j+b_j&=1\\
\sum_{j=1}^sa_j^{p+1}+(-1)^pb_j^{p+1}&=0
\end{align*}
\end{maththeorem}
Ohne das allgemeine Ergebnis zu verwenden wollen wir nun dennoch ein Verfahren $\Phi_{\deltat}^S$ mit Ordnung $p=2$ gewinnen. Dieses sollte möglichst den selben Zeitaufwand wie das Lie-Trotter-Splitting besitzen. Hierzu verwenden wir den Ansatz \[\Phi_{\deltat}^S=\Phi_{\deltathalf}^*\circ\Phi_{\deltathalf}\]
Wir erkennen sofort, dass es sich hierbei um ein symmetrisches Verfahren handelt. Als Komposition von Verfahren mit Ordnung 1 (siehe Satz \ref{lieorder1}) ist es ebenfalls mindestens von Ordnung 1. Als zusätzlich symmetrisches Verfahren gilt mit Satz \ref{adjfloworder}, dass das Verfahren mindestens von Ordnung 2 ist.\\
Dieses sogenannte \emph{Strang-Splitting} lässt sich unter Verwendung der aufgeteilten Flüsse (\ref{liesplittings}) auch darstellen als 

\noindent\begin{minipage}{0.3\textwidth}
\begin{tikzpicture}
  \matrix (m) [matrix of math nodes,row sep=3em,column sep=4em,minimum width=2em]
  {
  	 \, & y_{\frac{3}{4}} & y_1\\
     \, & y_{\frac{2}{4}} & \, \\
     y_0 & y_{\frac{1}{4}} & \,\\};
  \path[-stealth]
    (m-3-1.east|-m-3-2) edge node [below] {$\varphi_{\deltathalf}^A$} (m-3-2)
    (m-3-2) edge node [right] {$\varphi_{\deltathalf}^B$} (m-2-2)
    (m-3-1) edge [dashed] node [above left] {$\Phi_{\deltat}$} (m-2-2)
    
    (m-1-2.east|-m-1-3) edge node [above] {$\varphi_{\deltathalf}^A$} (m-1-3)
    (m-2-2) edge node [left] {$\varphi_{\deltathalf}^B$} (m-1-2)
    (m-2-2) edge [dashed] node [below right] {$\Phi_{\deltat}^*$} (m-1-3);
\end{tikzpicture}
\end{minipage}%
\hfill%
\begin{minipage}{0.7\textwidth}
\begin{align}
\Phi_{\deltat}^S&=\varphi_{\deltathalf}^A\circ\varphi_{\deltathalf}^B\circ\varphi_{\deltathalf}^B\circ\varphi_{\deltathalf}^A\nonumber\\
&=\varphi_{\deltathalf}^A\circ\varphi_{\deltat}^B\circ\varphi_{\deltathalf}^A
\end{align}
\end{minipage}
Für einen Schritt ist also nur der anderthalbfache Aufwand nötig wie beim Lie-Trotter-Splitting, aber man gewinnt bereits eine Ordnung dazu. Unter Umständen kann dies sogar noch verbessert werden:\\
Gegeben sei ein Anfangswert $y_0$ für das System (\ref{orddgl}). Ist man an einer Approximation $y_T$ der Lösung $y(T)$ zu einem Zeitpunkt $T=n\deltat>0$ interessiert, so gilt unter Verwendung des Strang-Splittingverfahrens
\begin{align*}
y_T&=\underbrace{\Phi_{\deltat}^S\circ\dots\circ\Phi_{\deltat}^S}_{n\text{ mal}}(y_0)\\
&=\left(\varphi_{\deltathalf}^A\circ\varphi_{\deltat}^B\circ\varphi_{\deltathalf}^A\right)\circ\left(
\varphi_{\deltathalf}^A\circ\varphi_{\deltat}^B\circ\varphi_{\deltathalf}^A\right)\circ\dots\circ\left(
\varphi_{\deltathalf}^A\circ\varphi_{\deltat}^B\circ\varphi_{\deltathalf}^A\right)\\
&=\varphi_{\deltathalf}^A\circ\varphi_{\deltat}^B\circ\left(\varphi_{\deltat}^A\circ\varphi_{\deltat}^B\right)\circ\left(\varphi_{\deltat}^A\circ\varphi_{\deltat}^B\right)\circ\dots\circ\left(\varphi_{\deltat}^A\circ\varphi_{\deltat}^B\right)\circ\varphi_{\deltathalf}^A\\
&=\varphi_{\deltathalf}^A\circ\varphi_{\deltat}^B\circ
\underbrace{\Phi_{\deltat}^*\circ\dots\circ\Phi_{\deltat}^*}_{n-1\text{ mal}}\circ\,\varphi_{\deltathalf}^A
\end{align*}
Durch das Kombinieren der äußeren Halbschritte bekommt man also nun eine Darstellung des Strang-Splittings, die nur einen zusätzlichen Fluss berechnen muss und trotzdem Ordnung $p=2$ besitzt. Man muss hierbei allerdings auf Zwischenergebnisse verzichten, da diese eine weitere Anwendung von $\varphi_{\deltathalf}^A$ benötigen würden.\\[0.5cm]
Bisher sind wir davon ausgegangen, dass sich die Flüsse $\varphi_{\deltat}^A$ btw. $\varphi_{\deltat}^B$ exakt berechnen lassen. Dies ist in den seltensten Fällen der Fall und man verwendet stattdessen approximative Verfahren $\Phi_{\deltat}^A$ bzw. $\Phi_{\deltat}^B$. 
\begin{maththeorem}
Sind die Verfahren ebenfalls von Ordnung $p$, d.h. falls 
\begin{align*}
\Phi_{\deltat}^A(y(t))=\varphi_{\deltat}^A(y(t))+\mathcal{O}(\deltat^{p+1})\\
\Phi_{\deltat}^B(y(t))=\varphi_{\deltat}^B(y(t))+\mathcal{O}(\deltat^{p+1})
\end{align*}
so gelten Satz \ref{adjfloworder}, Satz \ref{lieorder1} und alle entsprechenden Aussagen aus Kapitel \ref{seclietrotter} und Kapitel \ref{secstrang} für $\Phi_{\deltat}^A$ und $\Phi_{\deltat}^B$ anstelle von $\varphi_{\deltat}^A$ und $\varphi_{\deltat}^B$.\\
Insbesondere ist das Strang-Splitting immer noch von Ordnung 2, wenn alle verwendeten Approximationen von Ordnung 2 sind.
\end{maththeorem}
\begin{proof}[Beweisidee]
Am Beispiel von $\varphi_{\deltat}^A$ und $p=1$:\\
Man verwendet die Darstellung $\varphi_{\deltat}^A(y_0)=y_0+\deltat f^A(y_0)+\mathcal{O}(\deltat^2)$ und ersetzt $\varphi_{\deltat}^A$ durch $\Phi_{\deltat}^A$. Die Vorgehensweise in den Beweisen muss dann höchstens minimal angepasst werden.\\
Man beachte hierbei, dass nun $\Phi_{\deltat}^A$ im Gegensatz zu $\varphi_{\deltat}^A$ nicht mehr symmetrisch ist. Die Argumentation über die Ordnung von symmetrischen Verfahren kann also nicht direkt verwendet werden, um $p=2$ für das Strang-Splittings zu begründen. Außerdem kann dies die Möglichkeit, das Strang-Splitting durch Kombination der Halbschritte zu beschleunigen, zunichte machen.
\end{proof}

\section{Erster Einblick in das Polynomial Chaos}
\todo[inline]{Diese Einführung erst nachdem wir kurz gesagt haben was Polynomial Chaos eigentlich ist. Vllt auch etwas kürzen? So versteht das jemand beim ersten Mal lesen nicht.}
Die Entstehung des \emph{general Polynomial Chaos}(gPC) ist eine Geschichte über den Versuch, stochastische Abhängigkeit und Integrationstheorie zu kombinieren. Sie wurzelte in der Arbeit von Wiener (\autocite{norbertwiener1938}), wo er das \emph{homogene Chaos} einführte, um die Entwicklung von Unsicherheit in einem dynamischen, chaotischen, physikalischen System zu konkretisieren. Dies erklärt auch den Ursprung der Bezeichnung "`Chaos"' im Kontext der chaotischen Brownschen Bewegung im Wiener-Prozess.\\
Ghanem verwendete dann Hermite Polynome um mithilfe dieser Orthogonalbasis gaußsche Prozesse darzustellen. Dies wendete er erfolgreich auf einige Probleme aus den Ingenieurswissenschaften an, ein Überblick wird in (\autocite{GhaSpa91}) gegeben.\\
Xiu verallgemeinerte in seiner Arbeit (\autocite{xiu2002}) diesen Ansatz auf andere Kombinationen aus Stochastischer Verteilung und Polynombasis. Gemäß dem Askey Schema (siehe Abbildung \ref{askeyscheme}) werden Beziehungen einiger bekannter Polynom Orthonormalbasen verdeutlicht. Man beobachtet außerdem, dass einige Gewichte den Dichtefunktionen von bekannten Verteilungen entsprechen und möchte nun zu einer gegebenen Verteilung die bestmögliche Polynombasis verwenden. An Modellproblemen demonstrierte Xiu jeweils optimale exponentielle Konvergenz für die verschiedenen Paarungen.\\
\begin{figure}
\center
\includegraphics[width=0.8\linewidth]{Figures/askeyscheme.png}
\caption{Askey Schema; Ein Pfeil symbolisiert den Übergang einer Basis in eine andere durch Grenzwertbildung eines Parameters; Quelle: \autocite{webaskey}}
\label{askeyscheme}
\end{figure}
\begin{mathbsp}[Grenzwertbildung im Askey Schema]
Es gilt für die Jacobi Polynome $P_n^{(\alpha,\beta)}(x)$ und die Hermite Polynome $H_n(x)$ die Beziehung
\[\lim\limits_{\alpha\to\infty}\alpha^{-\onehalf n}P_n^{(\alpha,\alpha)}\left(\frac{x}{\sqrt{\alpha}}\right)=\frac{H_n(x)}{2^nn!}\]
Für eine genauere Betrachtung der Beziehungen der hypergeometrischen Polynome im Askey Schema sei auf \autocite{koekoekswart98} verwiesen.
\end{mathbsp}
Im Mittelpunkt der Theorie des gPC stehen orthgonale Polynombasen. Die Verflechtung mit der Stochastik geschieht dann auf natürliche Weise durch die passende Wahl des zugrunde liegenden Skalarproduktes.
\subsection{Orthogonale Polynome}
Wir beginnen mit grundlegender Notation und Namensgebung um das zentrale Werkzeug der Polynomapproximation festzulegen. Später werden diese Polynome verwendet, um die Zufallsabhängigkeit der auftretenden Funktionen darzustellen, indem das Orthogonalitätsmaß der Dichtefunktion entspricht.
\begin{mathdef}
Ein allgemeines Polynom $Q$ vom Grad $n$ lässt sich darstellen als
\[Q_n(x)=q_nx^n+q_{n-1}^{n-1}+\dots+q_1x+q_0,\quad q_n\ne 0\]
Wir bezeichnen mit 
\[\frac{Q_n(x)}{q_n}\]
die normierte (engl. \emph{monic}) Version des Polynoms $Q$, also mit führendem Koeffizienten gleich 1.\\
 Ein System $\left\lbrace Q_n(x),\: n\in\mathcal{N}\right\rbrace$ von Polynomen mit $\mathcal{N}=\N_0=\N\cup\lbrace 0\rbrace$ oder $\mathcal{N}=\lbrace 0,1,\dots,N-1,N\rbrace$ heißt Orthogonalsystem, wenn bezüglich eines reellen positiven Maßes $\alpha$ die Orthogonalitätsbedingung
 \[\langle Q_n(x),Q_m(x)\rangle =\int_S Q_n(x)Q_m(x)d\alpha(x)=\gamma_n\delta_{mn},\quad m,n\in\mathcal{N}\]
 erfüllt ist. Dabei ist $\delta_{mn}$ das Kronecker-Delta, $S$ der nicht notwendigerweise endliche Träger des Maßes $\alpha$ und $y_n>0$ die Normalisierungskonstanten. Es gilt 
 \[\gamma_n=\int_S Q_n^2(x)d\alpha(x),\quad n\in\mathcal{N}\] und O.B.d.A. werden wir in folgenden Kapiteln stets von $\gamma_n=1$ ausgehen. In diesem Fall nennt man das System ein Orthonormalsystem.
\end{mathdef}
Das Hermite Chaos ist ein motivierendes Beispiel dafür, dass solche Verflechtungen von Polynombasis und stochastischer Verteilung natürlicherweise existieren.
\begin{mathbsp}[Hermite Chaos]
Ist $S=(-\infty,\infty)$, $w(x)=\frac{1}{\sqrt{2\pi}}e^{-\frac{x^2}{2}}$ die Dichtefunktion des Gauss-Verteilung und verwenden wir diese als Dichte des Maßes $\alpha$, so sind die Hermite Polynome
\[H_0(x)\equiv 1,\quad H_1(x)=x,\quad H_2(x)=x^2-1,\quad H_n(x)=xH_{n-1}(x)-(n-1)H_{n-2}(x)\]
ein Orthogonalsystem bezüglich $\langle\cdot,\cdot\rangle_{L_w^2(S)}$ mit Normalisierungskonstanten $\gamma_n=n!$.\\
Der Erwartungswert der Gauss-Verteilung
\[\E_w[Z]= \int_{-\infty}^\infty Z(x)w(x)dx\] erfüllt also die entscheidende Beziehung des Polynomial Chaos
\[\langle H_n,H_m\rangle_{L_w^2(S)} =\E_w \left[H_nH_m\right]=\gamma_n\delta_{nm}\]
Man beachte, dass diese Definition der Hermite Polynome leicht von der in der Literatur üblichen abweicht. Dies ist der Tatsache geschuldet, dass dort die entsprechende Gewichtsfunktion einen zusätzlichen Faktor $\sqrt{2}$ besitzt. Sie stellt somit keine Dichte einer Wahrscheinlichkeitsverteilung dar und ist für unsere Zwecke unpraktisch und wird nicht verwendet. 
\end{mathbsp}
Wir werden in dieser Arbeit auf Polynombasen, die diskreten Verteilungen entsprechen, verzichten. Es sei jedoch bemerkt, dass dies keine Einschränkung der Methode ist, sondern lediglich die Notation vereinfacht. Konkret werden wir uns mit Laguerre-, Jacobi-, Hermite- und Legendre-Polynomen beschäftigen, wobei letztere ein Spezialfall der Jacobipolynome darstellen. Details und Eigenschaften sind im Anhang \ref{AppendixA} zu finden.\\[0.3cm]
Eine wichtige Eigenschaft von Orthogonalsystemen von Polynomen ist eine besondere Art der rekursiven Darstellung, wie wir sie bereits bei den Hermite Polynomen gesehen haben.
\begin{maththeorem}[Drei Term Rekursion]
\label{threetermexist}
Ein Orthogonalsystem $\left\lbrace Q_n(x),\: n\in\N_0\right\rbrace$ von Polynomen erfüllt eine Drei-Term-Rekursion
\begin{eqnarray}
Q_n(x)=(a_nx+b_n)Q_{n-1}(x)-c_nQ_{n-2}(x),\quad n\in\N\\
Q_{-1}(x)\equiv 0, Q_0(x)\equiv 1\nonumber
\end{eqnarray}
wobei $a_n=\frac{k_n}{k_{n-1}}\neq 0$, $b_n\in\R$ und $c_n=\frac{a_n}{a_{n-1}}\cdot \frac{\gamma_n}{\gamma_{n-1}}$.
Dabei ist $k_n\neq 0$ der führende Faktor und $\gamma_n>0$ der Normalisierungsfaktor des Polynoms $Q_n$ . 
\end{maththeorem}
\begin{proof}
Es gilt $\text{deg}(Q_n)=n,\forall n\in\N_0$. Für $a_n=\frac{k_n}{k_{n-1}}$ ist $Q_n(x)-a_nxQ_{n-1}(x)$ ein Polynom vom Grad $\le n-1$. Dieses lässt sich also schreiben als
\[Q_n(x)-a_nxQ_{n-1}(x)=\sum_{m=0}^{n-1}r_mQ_m(x)\]
Für $k\le n-1$ folgt also aus der Orthogonalität bezüglich des Skalarproduktes $\langle\cdot,\cdot\rangle$
\begin{equation*}
\langle Q_n(x)-a_nxQ_{n-1}(x),Q_k(x)\rangle=\sum_{m=0}^{n-1}r_m\langle Q_m(x),Q_k(x)\rangle=r_k\langle Q_k(x),Q_k(x)\rangle=r_k\gamma_k
\end{equation*}
und durch Verwenden von $\langle Q_n(x),Q_k(x)\rangle = 0$ und Umstellen
\[ -a_n\langle Q_{n-1}(x),xQ_k(x)\rangle=r_k\gamma_k\]
Somit folgt für $k<n-2$ wegen $\text{deg}(xQ_k(x))<n-1$ dass $r_k=0$. Damit erfüllen die Polynome eine Drei-Term-Rekursion
\[Q_n(x)-a_nxQ_{n-1}(x)=r_{n-1}Q_{n-1}(x)+r_{n-2}Q_{n-2}(x)\]
Nun definieren wir $b_n=r_{n-1}$, $c_n=-r_{n-2}$ und berechnen schlussendlich
\[r_{n-2}\gamma_{n-2}=\langle -a_nxQ_{n-1}(x),Q_{n-2}(x)\rangle =-a_n\langle Q_{n-1}(x),xQ_{n-2}(x)\rangle=-a_n\frac{k_{n-2}}{k_{n-1}}\gamma_{n-1}\]
\end{proof}
Diese und noch viele weitere Eigenschaften von den verschiedenen Orthogonalsystemen von Polynomen finden sich in \autocite{weborthopoly} und in vielen weiteren Werken über Polynome. Beispielsweise lässt sich auch die Umkehrung zeigen (unter milden Anforderungen an die Koeffizienten): Zu jeder durch eine Drei-Term-Rekursion definierte Menge an Polynomen existiert ein Maß $\alpha$, bezüglich dessen die Polynome ein Orthogonalsystem bilden (Satz von Favard).\\[0.3cm]
Die auftretenden Integrale in den Skalarprodukten werden wir später mit Gauss-Quadraturformeln approximieren. An dieser und an anderen Stellen werden wir später die Nullstellen der Polynome als Stützstellen benötigen. Basierend auf den Koeffizienten der Drei-Term-Rekursion entwickelten Golub und Welsch in \autocite{GolubWelsch} einen effizienten stabilen Algorithmus, der für die Berechnung der Nullstellen mit der Berechnung der Eigenwerte einer symmetrischen Tridiagonalmatrix auskommt.

\begin{maththeorem}[Golub Welsch Algorithmus]
\label{golubwelschalg}
Ist $\left\lbrace Q_i(x),\: i=0,\dots,n\right\rbrace$ ein Orthogonalsystem von Polynomen welches die Drei-Term-Rekursion
\begin{eqnarray}
\label{threetermgolub}
Q_i(x)=(a_ix+b_i)Q_{i-1}(x)-c_iQ_{i-2}(x),\quad i=1,\dots , n\\
Q_{-1}(x)\equiv 0, Q_0(x)\equiv 1\nonumber
\end{eqnarray}
erfüllt für $a_i>0,c_i>0$, so gilt:
\begin{align*}
Q_n(x_j)=0\: \equivalent \: &x_j\text{ ist Eigenwert von }J=\begin{pmatrix}
\alpha_1 & \beta_1 &  &  &  \\ 
\beta_1 & \alpha_2 & \beta_2 & & \text{\huge0} \\ 
 & \beta_2 & \ddots & \ddots &  \\ 
 &  & \ddots & & \beta_{n-1} \\
\text{\huge0} &  &  & \beta_{n-1} & \alpha_n
\end{pmatrix},\quad j=1,\dots,n\\
&\text{ mit } \alpha_i=-\frac{b_i}{a_i},\quad \beta_i=\sqrt{\frac{c_{i+1}}{a_ia_{i+1}}},\quad i=1,\dots,n
\end{align*}
Die Eigenwerte der symmetrischen Tridiagonalmatrix sind reell und lassen sich mit spezialisierten Algorithmen effizient berechnen. Insbesondere sind alle Nullstellen des Polynoms $Q_n$ reell.
\end{maththeorem}
\begin{proof}
Zuerst schreiben wir die Rekursionen (\ref{threetermgolub}) für $i=1,\dots,n$ in Matrixform und teilen zeilenweise durch $a_i$ und sortieren den Term $xQ_{i-1}$ auf die linke Seite
\begin{equation*}
x\cdot
\underbrace{\begin{pmatrix}Q_0(x)\\ Q_1(x) \\ \vdots \\ \\ \vdots \\ Q_{n-1}(x)\end{pmatrix}}_{=Q(x)}
=\underbrace{\begin{pmatrix}
-\frac{b_1}{a_1} & \frac{1}{a_1} &  &  &  \\ 
\frac{c_2}{a_2} & -\frac{b_2}{a_2} & \frac{1}{a_2} & & \text{\huge0} \\ 
 & \frac{c_3}{a_3} & \ddots & \ddots &  \\ 
 &  & \ddots & & \frac{1}{a_{n-1}} \\
\text{\huge0} &  &  & \frac{c_n}{a_n} & -\frac{b_n}{a_n}
\end{pmatrix}}_{=T}
\begin{pmatrix}Q_0(x)\\ Q_1(x) \\ \vdots \\ \\ \vdots \\ Q_{n-1}(x)\end{pmatrix} 
+\begin{pmatrix}0 \\ 0 \\ \vdots \\ \vdots \\ 0 \\ \frac{Q_{n}(x)}{a_n}\end{pmatrix} 
\end{equation*}
In kompakter Form lässt sich dies darstellen als
\[xQ(x)=TQ(x)+\frac{1}{a_n}Q_n(x)e_n\]
wobei $e_n=(0,\dots,0,1)^T$ der n-te Einheitsvektor ist. Hieran erkennt man sofort, dass $Q_n(x_j)=0\equivalent x_j\text{ ist Eigenwert von }T$.\\
Weiterhin erhalten wir aus Satz \ref{threetermexist}, dass $\frac{c_i}{a_i}=\frac{1}{a_{i-1}}\frac{\gamma_i}{\gamma_{i-1}}$ und somit wäre $T=J$ bereits symmetrisch, wenn die $Q_i$ eine Orthonormalbasis bilden.\\
Andernfalls führen wir eine Äquivalenztransformation durch mit der Diagonalmatrix $D=\text{diag}\left((d_1,\dots,d_n)^T\right)$, so dass $J=DTD^{-1}$. Koeffizientenvergleich ergibt die Bedingung $d_{i+1}=d_i\sqrt{\frac{a_{i+1}}{c_{i+1}a_i}}$; mit einer möglichen Wahl $d_1=1$ des freien Parameters $d_1$ erhalten wir somit die gewünschte symmetrische Matrix.
\end{proof}
Wir bemerken zusätzlich ohne Beweis, dass das Polynom $Q_n$ für $n\ge 1$ insgesamt $n$ verschiedene reelle Nullstellen besitzt. Dies ist eine fundamentale Aussage über Orthogonalpolynome und lässt sich induktiv einsehen unter Ausnutzung von $n$ Vorzeichenwechseln, die durch die Orthogonalität vorhanden sein müssen.

\subsection{Polynomiale Approximationstheorie}
Es gibt sehr viele Ergebnisse über die Approximation von Funktionen $f\colon \R\to\R$ durch Polynome $p\in\Poly_n$. Um ein Gefühl für die Mächtigkeit --aber auch für die Grenzen-- von polynomialer Approximation zu bekommen, werden an dieser Stelle einige zentrale Sätze genannt. Auf Beweise werden wir größtenteils verzichten, jedoch basierend auf \autocite{Trefethen} eine Diskussion der Interpretation dieser Ergebnisse aus dem numerischen Blickwinkel der Interpolation und Quadratur geben.
\begin{maththeorem}[Weierstrass]
Ist $I$ ein beschränktes Intervall und $f\in C^0(\bar{I})$ stetig, dann gibt es für jedes $\epsilon >0$ ein $n\in\N$ und $p\in\Poly_n$, so dass
\[|f(x)-p(x)|<\epsilon,\quad \forall x\in\bar{I}\]
\end{maththeorem}
Folglich gilt für die \emph{gleichmäßige Bestapproximation} $\phi_n$, welche zu $f\in C^0(\bar{I})$ eine eindeutige Lösung $\phi_n(f)\in\Poly_n$ mit minimaler Maximumsnorm ergibt, die Aussage
\[\lim\limits_{n\to\infty}\norm{f-\phi_n(f)}_\infty=0\] 
Dieses Ergebnis sieht sehr vielversprechend aus, jedoch wurde von Faber in \autocite{faber14} gezeigt, dass es kein Interpolationsschema gibt, das für alle stetigen Funktionen auf diese Weise konvergiert. Doch dieser vermeintliche Rückschlag ist in der Praxis selten von Bedeutung! Problematisch sind Interpolationsschemata, die beispielsweise auf gleichverteilten Knotenpunkten basieren und dann unter dem Runge-Phänomen leiden, welches bei Funktionen wie $f(x)=\frac{1}{1+x^2}$ an den Rändern für eine schlechte Approximationsgüte sorgt. Erfüllt $f$ jedoch minimale Glattheitsanforderungen (wie bereits Lipschitzstetigkeit) und verwendet man Chebyshev Interpolationspunkte, so ist die Approximationsgüte bereits bereits für kleines $n$ gesichert.\\[0.3cm]
Im Folgenden lösen wir uns von dem beschränkten Intervall $I$ und lassen auch $I=\R$ oder $I=[0,\infty)$ zu. Außerdem wird die Bestapproximation bezüglich der Maximumsnorm später von wenig Nutzen sein. Daher führen wir zuerst einen neuen normierten Raum ein:
\begin{mathdef}
Sei $w\colon I\to\R_{>0}$ ein Gewicht, dann definieren wir den gewichteten $L^2$ Raum als
\begin{equation*}
L_w^2(I)\coloneqq \left\lbrace v\colon I\to\R | \int_Iv^2(x)w(x)dx<\infty\right\rbrace
\end{equation*}
mit dem inneren Produkt 
\[\langle u,v\rangle_{L_w^2(I)}=\int_I u(x)v(x)w(x)dx,\quad u,v\in L_w^2(I)\]
und zugehöriger Norm
\[\norm{u}_{L_w^2(I)}=\left(\int_I u^2(x)w(x)dx\right)^{\onehalf}\]
\end{mathdef}
Der Begriff der Bestapproximation lässt sich auch auf diesen neuen Raum erweitern. Dazu verwenden wir eine orthogonale Projektion auf eine orthogonale Polynombasis und erhalten eine einfache Darstellung des Bestapproximationsoperators.
\begin{maththeorem}
Ist $N\in\N_0$ und $\lbrace Q_k(x)\rbrace_{k=0}^N\subset \Poly_N$ ein Orthogonalsystem bezüglich des positiven Gewichtes $w(\cdot)$, also 
\[\langle Q_m(x),Q_n(x)\rangle_{L_w^2(I)}=\norm{Q_m}^2_{L_w^2(I)}\delta_{m,n},\quad, 0\le m,n\le N\]
so gilt für den Projektionsoperator $P_N\colon L_w^2(I)\to \Poly_N$ definiert durch 
\[P_Nf\coloneqq \sum_{k=0}^N\hat{f}_kQ_k(x),\quad \hat{f}_k\coloneqq \frac{1}{\norm{Q_k}_{L_w^2(I)}^2}\langle f,Q_k\rangle_{L_w^2(I)},\qquad f\in L^2_w(I),\]
die Bestapproximationsaussage
\[\norm{f-P_Nf}_{L_w^2(I)}=\inf\limits_{\psi\in\Poly_N}\norm{f-\psi}_{L_w^2(I)}\]
\end{maththeorem}
Dies entspricht der Fourier-Entwicklung im Hilbertraum. Entsprechend nennt man die Koeffizienten $\hat{f}_k$ die (verallgemeinerten) Fourier Koeffizienten. Die Projektion ist eine Orthogonalprojektion in dem Sinne, dass der Fehler $f-P_Nf$ senkrecht zum Polynomraum $\Poly_N$ ist.\\
Man beachte, dass der Name "`Bestapproximation"' sich strikt auf die verwendete Norm bezieht. Punktweise Aussagen lassen sich somit im Allgemeinen nicht treffen. 
\begin{mathbsp}
Als Beispiel vergleichen wir die Bestapproximation bezüglich der Maximumsnorm und die Chebyshev Interpolation vom Grad $N=100$ von der Funktion $f(x)=|x-\frac{1}{4}|$ in $[-1,1]$. Wie wir an Abbildung \ref{polyapproxcomp} erkennen, ist der maximale Fehler der Bestapproximation zwar geringer, aber tritt an $N+2$ Punkten auf, wohingegen der Fehler der Chebyshev Fehler zu den Rändern hin stark abnimmt. Dies soll verdeutlichen, dass die Bestapproximation nicht immer in jedem Sinne die "`beste"' Approximation darstellt.
\begin{figure}[ht]
 \center
 \includegraphics[width=\linewidth]{Figures/polynomial_approx_comparison.png}
 \caption{Visualisierung der Fehler verschiedener Approximationen, angelehnt an \autocite{Trefethen}. Hierbei wurde die Bestapproximation mithilfe des \emph{Remez} Algorithmus aus dem Chebfun Projekt (http://www.chebfun.org/) berechnet. Der Fehler der Chebyshev Approximation nimmt zu den Rändern hin stark ab.}
 \label{polyapproxcomp}
\end{figure}
\end{mathbsp}
Wie schon bei der Bestapproximation bzgl. der Maximumsnorm erhält man auch für die verallgemeinerte eine Konvergenzaussage. Für unbeschränktes $I$ sind die Beweise sehr aufwändig und beispielsweise in \autocite{CouHil53} zu finden. 
\begin{maththeorem}
Für jedes $f\in L_w^2(I)$ gilt
\[\lim\limits_{N\to\infty}\norm{f-P_Nf}_{L_w^2(I)}=0\]
\end{maththeorem}
Die Konvergenzordnung ist hierbei abhängig von der Regularität von $f$ und der Art der für die Projektion verwendeten Orthogonalpolynome. Man nennt diese Art der Konvergenz in der Literatur auch \emph{Spektrale Konvergenz}, da die Konvergenzrate abhängig von der Glattheit der Funktion ist. Man muss also aufpassen eine hohe Approximationsordnung nicht mit einer hohen Approximationsgenauigkeit zu verwechseln.\\
Für das Beispiel der Legendre Polynome, also mit konstanter Gewichtsfunktion $w$, zeigt Xiu in \autocite[Theorem 3.6]{dongbinxiu2010} folgendes Ergebnis
\begin{maththeorem}
\label{spectralconvth}
Für jedes $f\in H_w^p[-1,1]\coloneqq \left\lbrace v\colon I\to\R | \frac{d^mv}{dx^m}\in L_w^2(I),0\le m\le p\right\rbrace$ mit $p\ge 0, w\equiv 1$ gibt es eine Konstante $C$ unabhängig von $N$, so dass
\[\norm{f-P_Nf}_{L_w^2[-1,1]}\le CN^{-p}\norm{f}_{H_w^p[-1,1]}\]
\end{maththeorem}
\begin{mathbsp}
Als Veranschaulichung von Satz \ref{spectralconvth} approximieren wir verschiedene Funktionen $f\colon [-1,1]\to\R$ durch die orthogonale Projektion $P_Nf$ für steigendes $N\in\N$. In Abbildung \ref{figurespectralconverg} sind dafür die Fehler der Approximationen dargestellt. Dabei sind die $f$ unterschiedlich glatt und man erkennt unterschiedliche Konvergenzordnungen. Insbesondere erahnt man für die analytische Funktion $f(x)=\sin(\pi x)$ exponentielle Konvergenz in $N$. Da der General-Purpose Integrator, der für die Berechnung der Fourierkoeffizienten verwendet wurde für zu hohe Polynomgrade versagt, bricht die Darstellung an der entsprechenden Stelle ab.
\begin{figure}[h]
\includegraphics[width=\textwidth]{Figures/spectral_convergence_legendre.png}
\caption{Spektrale Konvergenz von Funktionen verschiedener Glattheit mithilfe von Legendre Polynomen und Gewichtsfunktion $w\equiv \onehalf$ auf $[-1,1]$.}
\label{figurespectralconverg}
\end{figure}
\end{mathbsp}
